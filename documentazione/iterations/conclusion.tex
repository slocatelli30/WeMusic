\section{Sviluppi futuri}
Alcuni dei casi d'uso previsti originariamente 
non sono ancora stati completamente implementati, come era stato 
inizialmente pianificato. Tuttavia, il processo di sviluppo 
è ancora in corso e l'obiettivo è raggiungere una copertura 
completa dei casi d'uso identificati.

Di seguito è riportata una tabella riassuntiva che elenca 
i casi d'uso già implementati e quelli ancora da implementare 
nelle future fasi di sviluppo. Questo elenco mira a fornire 
una panoramica chiara delle attività svolte e delle attività 
rimanenti, consentendo una migliore pianificazione anche grazie al tipo
di priorità assegnato al caso d'uso in questione. 

I casi d'uso che non sono stati implementati si riferiscono 
specificamente alle funzionalità che sono attive solo dal lato 
amministratore e non sono ancora disponibili per gli utenti regolari. 
Anche se queste funzionalità possono essere funzionanti e operative nel 
contesto amministrativo, non possono ancora essere considerate come 
completamente implementate fino a quando non saranno accessibili e utilizzabili 
anche dagli utenti standard. 

Pertanto, è importante distinguere tra le funzionalità 
che sono attualmente disponibili solo per gli amministratori e quelle che sono 
pienamente accessibili e utilizzabili dagli utenti regolari.

\newpage
\subsection{Casi d'uso implementati}
\begin{table} [h!]
    \begin{center}
        \begin{tabular}{ |Sc|Sc|Sc| } 
         \hline
         Codice & Caso d'uso & Priorità \\ 
         \hline
         \hline
         \texttt{UC2} & \texttt{Sign in} & \texttt{Alta} \\ 
         \hline
         \texttt{UC3} & \texttt{Sign out} & \texttt{Alta} \\ 
         \hline
         \texttt{UC4} & \texttt{Cerca Brano} & \texttt{Alta} \\ 
         \hline
         \texttt{UC5} & \texttt{Cerca Album} & \texttt{Media} \\ 
         \hline
         \texttt{UC6} & \texttt{Cerca Artista} & \texttt{Media} \\ 
         \hline
         \texttt{UC7} & \texttt{Scarica Brano} & \texttt{Alta} \\ 
         \hline
         \texttt{UC8} & \texttt{Like al brano} & \texttt{Alta} \\ 
         \hline
         \texttt{UC9} & \texttt{Aggiungi brano a Playlist} & \texttt{Alta} \\ 
         \hline
         \texttt{UC15} & \texttt{Visualizza Playlist} & \texttt{Alta} \\ 
         \hline
         \texttt{UC16} & \texttt{Crea nuova Playlist} & \texttt{Alta} \\ 
         \hline
         \texttt{UC17} & \texttt{Elimina Playlist} & \texttt{Alta} \\ 
         \hline
         \texttt{UC18} & \texttt{Modifica Playlist} & \texttt{Media} \\ 
         \hline
         \texttt{UC10} & \texttt{Cerca Utente} & \texttt{Alta} \\ 
         \hline
         \texttt{UC11} & \texttt{Aggiungi Utente} & \texttt{Alta} \\ 
         \hline
         \texttt{UC12} & \texttt{Visualizza informazioni profilo} & \texttt{Media} \\ 
         \hline
         \texttt{UC19} & \texttt{Discover} & \texttt{Alta} \\ 
         \hline
        \end{tabular}
    \end{center}
    \caption{Tabella UC implementati}
\end{table}
    
\newpage
\subsection{Casi d'uso non implementati} 
\begin{table} [h!]
    \begin{center}
        \begin{tabular}{ |Sc|Sc|Sc| } 
         \hline
         Codice & Caso d'uso & Priorità \\ 
         \hline
         \hline
         \texttt{UC1} & \texttt{Sign up} & \texttt{Alta} \\ 
         \hline
         \texttt{UC13} & \texttt{Modifica profilo} & \texttt{Media} \\ 
         \hline
         \texttt{UC14} & \texttt{Elimina profilo} & \texttt{Alta} \\ 
         \hline
         \texttt{UC20} & \texttt{Crea pagina Artista} & \texttt{Alta} \\ 
         \hline
         \texttt{UC21} & \texttt{Visualizza pagina Artista} & \texttt{Alta} \\ 
         \hline
         \texttt{UC22} & \texttt{Aggiungi Brano} & \texttt{Alta} \\ 
         \hline
         \texttt{UC23} & \texttt{Aggiungi Album} & \texttt{Media} \\ 
         \hline
         \texttt{UC24} & \texttt{Personalizza pagina Artista} & \texttt{Media} \\ 
         \hline
         \texttt{UC25} & \texttt{Consulta anagrafica} & \texttt{Media} \\ 
         \hline
         \end{tabular}
    \end{center}
    \caption{Tabella UC non implementati}
\end{table}

\newpage
\section{Toolchain e tecnologie utilizzate}
La seguente sezione comprende tutte le scelte di librerie, framework e piattaforme
che sono state effettuate per l'implementazione del progetto.

\subsection{Modellazione}
\begin{itemize}
      \item \textbf{Diagrammi UML di casi d'uso, deployment, delle classi:} Star UML,  strumento di modellazione UML che 
      consente agli sviluppatori e agli analisti di creare modelli visivi delle loro applicazioni software; supporta anche 
      l'importazione ed esportazione di file in vari formati UML, facilitando la collaborazione e lo scambio di modelli.
      con altri strumenti di modellazione UML.
      \item \textbf{Diagrammi UML di componenti, attività:}  Draw.io, uno strumento versatile che può essere utilizzato 
      sia online che offline ed è disponibile come open source. È utilizzato per creare una vasta gamma di diagrammi generici e,
      oltre a ciò, offre anche strumenti specifici per la modellazione UML.
\end{itemize}

\vspace{5pt}
\subsection{Linguaggi e librerie}

\subsubsection{Backend} Per l'implementazione del back-end sono stati usati:
\begin{itemize}
    \item \textbf{Python} come linguaggio di programmazione, scelto per la sua semplicità
    e portabilità;
    \item \textbf{PIP} come package manager per Python;
    \item \textbf{Django}, un framework Python per la realizzazione di server, gestione
    di database e templating;
    \item \textbf{SQLite3} come DBMS, un dialetto SQL che offre un'ottima integrazione
    con Python; pur non essendo il più performante, è stato scelto per
    rapidità di implementazione, e in iterazioni future è possibile scambiarlo
    con altri DBMS più performanti;
    \item \textbf{Pylint} per l'analisi statica del codice Python, l'identificazione realtime
    di errori, warning e la fornitura di ottimizzazioni e suggerimenti;
    \item \textbf{Django Test Framework} e \textbf{Coverage.py} per l'analisi dinamica del
    codice Python, dei quali il primo è integrato in Django, mentre il secondo
    è uno strumento esterno per misurare la copertura del codice che
    fornisce anche un output interattivo in HTML.
\end{itemize}

\subsubsection{Frontend} Per l'implementazione del front-end sono stati usati:
\begin{itemize}
    \item \textbf{Bootstrap}, un framework che comprende svariati stili di default per
    velocizzare lo sviluppo di una UI coerente e reattiva;
    \item \textbf{HTML} e \textbf{CSS} come linguaggi di markup e styling delle pagine web;
    \item \textbf{Electron}, una libreria usata per poter compilare il codice 
     in una app cross-platform.
\end{itemize}

\vspace{5pt}
\subsection{Organizzazione del team, Documentazione e Versioning}
L'organizzazione del team è avvenuta tramite \textbf{Discord}.
La documentazione è stata scritta in \textbf{Latex}, compilata tramite \textbf{Visual Studio Code}.
Il sistema di versioning utilizzato è \textbf{Git}, un sistema di versioning distribuito,
open source e gratuito. Il codice del progetto è presente anche su \textbf{GitHub}, un sito per remote hosting
di progetti git.


\vspace{5pt}
\subsection{Ambienti e IDEs}
Lo strumento principale utilizzato è \textbf{Visual Studio Code}, un editor di
testo open source e gratuito realizzato per i sistemi operativi Linux, macOS
e Windows.

Esso non è un IDE, in quanto non offre funzionalità di build automation
e debugging, ma si basa invece su un sistema di estensioni, sia per syntax
highlighting, linting, testing e building, mentre tutti gli strumenti per il
funzionamento dei linguaggi sono locali e forniti dal sistema operativo.

Fra le molte funzionalità, è presente anche la \textit{condivisione live} del codice, in
modo da poterne discutere in real-time e poter usare tecniche di sviluppo
agili che richiedono più persone, come il \textit{pair programming}, anche a distanza.


\section{Analisi del software}
\subsection{Requisiti non funzionali}
Le tecnologie sopracitate nel paragrafo relativo alla Toolchain, offrono anche dei precisi
requisiti non funzionali, ovvero dei requisiti di qualità che definiscono le caratteristiche 
e le proprietà di un sistema software che vanno oltre la sua funzionalità specifica. 

Mentre i requisiti funzionali descrivono cosa fa il sistema, i requisiti non funzionali specificano come 
il sistema dovrebbe essere in termini di prestazioni, sicurezza, usabilità, affidabilità e altri aspetti.

Le tecnologie utilizzate quindi riescono ad offrire scalabilità, manutenibilità, prestazioni adeguate per 
applicazioni di piccole e medie dimensioni, portabilità multi-piattaforma, sicurezza avanzata e affidabilità 
grazie all'uso di framework consolidati e al supporto di una vasta comunità di sviluppatori.

Di seguito vengono analizzati più nel dettaglio:
\begin{itemize}
    \item \textbf{Scalabilità}
    Django, essendo un framework web robusto e scalabile, offre funzionalità di scalabilità 
    orizzontale e verticale. Può gestire un numero crescente di richieste e traffico senza 
    compromettere le prestazioni. Inoltre, SQLite può essere sostituito con un database più 
    scalabile come MySQL o PostgreSQL per gestire un carico più elevato.

    \item \textbf{Manutenibilità}
    Django segue il principio di "don't repeat yourself" (DRY) e promuove una struttura organizzata 
    del codice attraverso le sue convenzioni. Ciò rende più facile la manutenzione e l'estensione 
    dell'applicazione nel tempo. Inoltre, l'uso di Python come linguaggio di programmazione favorisce 
    una sintassi pulita e leggibile, facilitando la comprensione e la manutenzione del codice.

    \item \textbf{Prestazioni}
    L'utilizzo di Django e Python per il backend, insieme a SQLite come database leggero, offre prestazioni 
    efficienti per applicazioni di piccole e medie dimensioni. Tuttavia, per applicazioni ad alto carico o con 
    requisiti di prestazioni particolarmente elevati, è possibile sostituire SQLite con un database più performante 
    come MySQL o PostgreSQL.

    \item \textbf{Portabilità}
    Utilizzando Electron per creare un'applicazione desktop multi-piattaforma, è possibile rendere l'applicazione 
    compatibile con diversi sistemi operativi come Windows, macOS e Linux. Ciò consente agli utenti di utilizzare 
    l'applicazione indipendentemente dalla piattaforma scelta.

    \item \textbf{Sicurezza} 
    Django è noto per le sue robuste funzionalità di sicurezza. Offre meccanismi di autenticazione e autorizzazione integrati, 
    prevenzione di attacchi CSRF (Cross-Site Request Forgery) e protezione contro le vulnerabilità comuni come SQL 
    injection e XSS (Cross-Site Scripting). Tuttavia, è importante adottare le best practice di sicurezza durante lo sviluppo 
    e la configurazione dell'applicazione per garantire un livello adeguato di sicurezza.

    \item \textbf{Affidabilità}
    Django è ampiamente utilizzato e testato dalla comunità open source, il che contribuisce alla sua affidabilità. 
    Inoltre, l'uso di tecnologie consolidate come Django, Bootstrap e SQLite riduce il rischio di problemi imprevisti 
    e aiuta a fornire un'esperienza stabile agli utenti.

\end{itemize}

\subsection{Design Pattern}

\begin{itemize}
    \item \textbf{Abstraction (Astrazione)}
    Django utilizza l'astrazione attraverso i modelli per rappresentare concetti del dominio e le interazioni con il database.
    \item \textbf{Encapsulation (Incapsulamento)}
    Python e Django promuovono l'incapsulamento attraverso l'uso di classi e metodi per raggruppare dati e funzionalità correlate.
    \item \textbf{Information Hiding (Nascondere le informazioni)}
    Questo principio è supportato da Python e Django attraverso l'uso di attributi privati e metodi che nascondono gli aspetti interni dell'implementazione.
    \item \textbf{Separation of concerns (Separazione delle responsabilità)}
    Django segue il principio di separazione delle responsabilità attraverso l'architettura MVC, separando il modello, la vista e il controller.
    \item \textbf{Separation of interface and implementation (Separazione dell'interfaccia e dell'implementazione)} 
    Django supporta la separazione dell'interfaccia e dell'implementazione attraverso le view, che definiscono l'interfaccia utente, separate dal modello e dal controller.
    \item \textbf{Coupling and cohesion (Coupling e coesione)}
    Django promuove una bassa accoppiamento e alta coesione tra i componenti attraverso l'architettura MVC, il sistema di routing e l'organizzazione modulare delle applicazioni.
    \item \textbf{Single point of reference (Singolo punto di riferimento)}
    Django offre il concetto di URLconf, che funge da singolo punto di riferimento per le richieste HTTP e dirige il traffico verso le view appropriate.
    \item \textbf{Divide and conquer (Dividere e conquistare)}
    Questo principio di progettazione generale può essere applicato nel modo in cui viene organizzato il codice sia nel backend (Python e Django) che nel frontend (Bootstrap).


\end{itemize}

