\section{Introduzione}
Nella terza iterazione è stato approfondito è implementato l'algoritmo portante del sistema: l'algoritmo di Discover permette di "Scoprire" nuova 
musica e nuovi utenti in base alle preferenze dell'utente.


\section{Descrizione dell'algoritmo}
L'algoritmo di Discover è stato ideato per offrire all'utente un'esperienza completa di riproduzione e scoperta di nuova musica, unendo a ciò la 
componente social, permettendogli di conoscere nuove persone che condividono con l'utente i propri gusti musicali. Questo tipo di algoritmo 
tiene traccia dei generi preferiti dell'utente e lo utilizza per suggerire musica simile, ma non solo: allo stesso modo riesce a suggerire 
una lista di utenti che hanno gusti simili all'utente loggato, in modo da poterli aggiungere.  

I fattori tenuti in considerazione per lo sviluppon dell'algoritmo di Discover sono i seguenti: 
\begin{itemize}
    \item Il genere favorito dell'utente, ottenuto da un'analisi dei suoi brani preferiti;
    \item L'età dell'utente;
    \item Il sesso dell'utente.
\end{itemize}


Il funzionamento dell'algoritmo si basa su una previsione Machine Learning, tramite la libreria \textbf{Pandas} di Python;
Il nome "Pandas" fa riferimento sia a "Panel Data" che "Python Data Analysis", infatti è una libreria molto utilizzata per lavorare 
con dataset: tramite le sue molteplici funzioni consente di analizzare, pulire, esplorare e manipolare i dati, definendo 
conclusioni basate su teorie statiistiche. 
Come anticipato, in base al genere preferito dell'utente, il suo sesso e la sua età, l'algoritmo suggerisce dei brani ricavandoli 
da un dataset, filtrandoli secondo le variabili sopracitate. Al termine dell'esecuzione l'algoritmo mostrerà un elenco di brani suggeriti.

I passi base dell'algoritmo sono i seguenti: 

Inizialmente, viene creato un oggetto account perchè in seguito serviranno le informazioni dell'account corrent, passando request (parametro che 
mi da le info relative all'account che sto utilizzando, CIOè USER1), viene quindi ricavata l'età e il sesso dell'utente loggato (1=maschio, 0=femmina); 
in seguito utilizzo un training set (utenti.csv) contenente tutte le informazioni degli utenti: 
sesso, anni, genere musicale preferito. 
Le variabili di input X considerate sono le prime due, mentre la variabile di output Y è il genere musicale (ciò che suggerirà l'algoritmo),
dopo aver creato queste due tabelle istanziamo il modello tramite il comando DecisionTreeClassifier (pandas), 
dobbiamo poi fittare quindi il modello si allena sulle variabili scelte per darci in seguito il risultato. 
In seguito effettuo la predizione inserendo le informazioni relative all'utente loggeto (tramite modello.predict), indine 
effettuo e salvo la previsione relativa solo al genere musicale come campo. 


viene letto un training set sul quale la stima si allena, contenente delle info
su altri utenti (scrivere quali). In seguito viene istanziato il modello tramite la funzione
\textbf{DecisionTreeClassifier()} (scrivere cosa fa). 
In seguito il modello viene allenato sui valori e infine tramite la funzione 
\textbf{predict} viene effettuata una previsione del genere musicale in base alle variabili considerate. 


\subsection{Discover: Suggerimento brano}
Nella prima parte dell'algoritmo si effettua un filtro su tutte le canzoni presenti nel db e 
prenderle per il target che ci ha dato l'algoritmo, quindi seleziono tutte le canzoni presenti nel db che 
rispecchiano il genere suggerito dall'algoritmo, salvo in una lista. 
Quindi il risultato sarà

\subsection{Discover: Suggerimento amici}
La seconda parte dell'algoritmo è più meccanica: si basa sempre sul genere target individuato nella parte
iniziale dell'algoritmo, ma andiamo ad estrapolare una lista di amici suggeriti per l'utente. 
Inizialmente viene creata una lista amici vuota e salvo le info dell'utente loggato, prendo tutti 
i suoi amici in modo da escluderli in seguito nell'algoritmo(per evitare che vengano suggeriti quelli già amici), ed escludere
anche l'utente stesso. Dopo aver creato una lista degli identificativi, tramite un for each, scorro tutti gli utenti nel db
sfoglio le canzoni preferite da questi utenti nel db e salvo il genere di ogni canzone in una variabile y, 

Condizione per rientrare negli utenti suggeriti:
1. all'utente x (nel db) deve piacere almeno una canzone che abbia lo stesso genere target dell'algoritmo, 
2. non deve essere già amico dell'utente
3. non deve essere lo stesso utente loggato. 

Nel context poi passo la lista che ho creato. 

\subsection{Bootstrap}
Per il frontend è stato utilizzato Bootstrap is a free front-end framework for faster and easier web development
Bootstrap includes HTML and CSS based design templates for typography, forms, buttons, tables, navigation, modals, image carousels and many other, as well as optional JavaScript plugins
Bootstrap also gives you the ability to easily create responsive designs

vantaggi: 
Easy to use: Anybody with just basic knowledge of HTML and CSS can start using Bootstrap
Responsive features: Bootstrap's responsive CSS adjusts to phones, tablets, and desktops
Mobile-first approach: In Bootstrap 3, mobile-first styles are part of the core framework
Browser compatibility: Bootstrap is compatible with all modern browsers (Chrome, Firefox, Internet Explorer, Edge, Safari, and Opera)


\subsection{Flowchart algoritmo??}
