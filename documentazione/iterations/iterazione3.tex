\section{Introduzione}
Nella terza iterazione è stato approfondito è implementato l'algoritmo portante del sistema: l'algoritmo di Discover permette di "Scoprire" nuova 
musica e nuovi utenti in base alle preferenze dell'utente.


\section{Descrizione dell'algoritmo}
L'algoritmo di Discover è stato ideato per offrire all'utente un'esperienza completa di riproduzione e scoperta di nuova musica, unendo a ciò la 
componente social, permettendogli di conoscere nuove persone che condividono con l'utente i propri gusti musicali. Questo tipo di algoritmo 
tiene traccia dei generi preferiti dell'utente e lo utilizza per suggerire musica simile, ma non solo: allo stesso modo riesce a suggerire 
una lista di utenti che hanno gusti simili all'utente loggato, in modo da poterli aggiungere.  

I fattori tenuti in considerazione per lo sviluppon dell'algoritmo di Discover sono i seguenti: 
\begin{itemize}
    \item Il genere favorito dell'utente, ottenuto da un'analisi dei suoi brani preferiti;
    \item L'età dell'utente;
    \item Il sesso dell'utente.
\end{itemize}

Il funzionamento dell'algoritmo si basa su una previsione Machine Learning, tramite la libreria \textbf{Pandas} di Python;
Il nome "Pandas" fa riferimento sia a "Panel Data" che "Python Data Analysis", infatti è una libreria molto utilizzata per lavorare 
con dataset: tramite le sue molteplici funzioni consente di analizzare, pulire, esplorare e manipolare i dati, definendo 
conclusioni basate su teorie statiistiche. 
Come anticipato, in base al genere preferito dell'utente, il suo sesso e la sua età, l'algoritmo suggerisce dei brani ricavandoli 
da un dataset, filtrandoli secondo le variabili sopracitate. Al termine dell'esecuzione l'algoritmo mostrerà un elenco di brani suggeriti.

Inizialmente, viene creato un oggetto account contenente le informazioni relative all'account dell'utente loggato, che ci serviranno in seguito; 
viene passato il parametro \textbf{request} che è quello che contiene le informazioni di interesse, ovvero l'età e il sesso (1=maschio, 0=femmina). 
In seguito viene utilizzato un training set (\textbf{utenti.csv}) sul quale allenare la stima, contenente le informazioni degli utenti 
necessarie per estrapolare le variabili: quelle di input includono il sesso e l'età dell'utente, quella di output contiene solo 
il genere preferito (ciò che suggerirà l'algoritmo). 

Dopo aver creato queste due tabelle viene istanziato il modello previsionale tramite il comando \textbf{DecisionTreeClassifier}, un comando 
della libreria Pandas che permette, dandogli in input le variabili considerate, di fittare il modello che le lega. 

Viene quindi fittato il modello tramite il comando \textbf{modello.fit(...)}, che si allenerà sulle variabili scelte, età e sesso, per restituire 
il risultato (genere preferito). 

In seguito effettuo la predizione inserendo le informazioni relative all'utente loggato tramite la funzione \textbf{modello.predict}, ed infine 
salvo la previsione relativa solo alla variabile di genere musicale. 



\subsection{Discover: Suggerimento brano}
Nella prima parte dell'algoritmo si effettua un filtro su tutte le canzoni presenti nel db e 
prenderle per il target che ci ha dato l'algoritmo, quindi seleziono tutte le canzoni presenti nel db che 
rispecchiano il genere suggerito dall'algoritmo, salvo in una lista. 
Quindi il risultato sarà

\subsection{Discover: Suggerimento amici}
La seconda parte dell'algoritmo è più meccanica: si basa sempre sul genere target individuato nella parte
iniziale dell'algoritmo, ma andiamo ad estrapolare una lista di amici suggeriti per l'utente. 
Inizialmente viene creata una lista amici vuota e salvo le info dell'utente loggato, prendo tutti 
i suoi amici in modo da escluderli in seguito nell'algoritmo(per evitare che vengano suggeriti quelli già amici), ed escludere
anche l'utente stesso. Dopo aver creato una lista degli identificativi, tramite un for each, scorro tutti gli utenti nel db
sfoglio le canzoni preferite da questi utenti nel db e salvo il genere di ogni canzone in una variabile y, 

Condizione per rientrare negli utenti suggeriti:
1. all'utente x (nel db) deve piacere almeno una canzone che abbia lo stesso genere target dell'algoritmo, 
2. non deve essere già amico dell'utente
3. non deve essere lo stesso utente loggato. 

Nel context poi passo la lista che ho creato. 

\subsection{Bootstrap}
Per il frontend è stato utilizzato Bootstrap, un framework per frontend gratuito e per web development più semplice e veloce. 
Bootstrap include templates design basati su HTML e CSS per typografia, forme, pulsanti, tabelle, navigazione, caroselli di 
immagini e molto altro, come anche plugin opzionali si JavaScript, permettendo uno sviluppo semplice. 
E' molto semplice da usare e un alta compatibilità con tutti i browser maggiormente usati. 


\subsection{Flowchart algoritmo??}
I passi base dell'algoritmo sono i seguenti: 

